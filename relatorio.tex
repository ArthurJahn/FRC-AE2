
\documentclass[a4paper]{report} %padrao letterpaper, 10pt
\usepackage[utf8]{inputenc}
\usepackage[brazil]{babel}
\usepackage{amsfonts,amssymb,graphicx,enumerate}
\usepackage[centertags]{amsmath}
% Configuracoes de pagina
\usepackage[lmargin=3cm,rmargin=3cm,tmargin=3cm,bmargin=3cm]{geometry}
% Layout da pagina
\usepackage{hyperref}
% Layout para código
\usepackage{listings}
% Font for code style
\usepackage{courier}
% The 68 standard colors known to dvips
\usepackage[usenames,dvipsnames]{xcolor}
\hypersetup{pdfpagelayout=SinglePage, % ou TwoPageLeft
    colorlinks=true,
    pdftitle={Relatório de asuhaush Redes},
    pdfauthor={Áulus Diniz, Arthur Jahn}}

%*******************************************************
\definecolor{light-gray}{gray}{0.90}
\lstset{
  backgroundcolor=\color{light-gray},
  basicstyle=\footnotesize\ttfamily,
  breaklines=true
}

\title{Fundamentos de Redes de Computadores: \\ Camada de Transporte}
\author{Arthur Jahn, Áulus Diniz}
\date{26 de Outubro de 2015}    %para ocultar a data digite: \date{ }
%*******************************************************
\begin{document}    %Inicio do documento
\maketitle  %cria o titulo na capa

\tableofcontents %Sumario
%-------------------------------------------------------
\chapter{Protocolos de Camada de Aplicação}
\label{chap_primeiro} %rotulo que pode ser usado em referencia cruzada

\section{Monitoramento e Coleta de Pacotes}
\label{sec_primeiro_monitoramento}

	Para	se capturar os pacotes recebidos e enviados pela rede, foi utilizada a ferramenta \textit{tshark}, uma versão para terminal de comandos que possui todas as opções da ferramenta de interface gráfica \textit{WireShark}, ambas ferramentas de monitoramento e diagnóstico de redes.

	Nos experimentos efetuados para este trabalho, foi utilizada uma distribuição Linux Debian versão 7.9 e para tal, os seguintes passos foram necessários para instalação e configuração do ambiente de experimentação e documentação.
\begin{lstlisting}[language=bash]

	sudo aptitude install tshark

	sudo aptitude install texlive

	sudo aptitude install texlive-lang-portuguese


\end{lstlisting}

	Para que fosse possível analisar os pacotes específicos de cada protocolo da camada de aplicação, foi necessário analisar o tipo de pacote que deveria ser capturado segundo os protocolos de transporte utilizados. No caso de pacotes transmitidos pelo protocolo DNS da camada de aplicação, é necessário capturar apenas pacotes que utilizem na camada de transporte protocolo UDP e que possuam 53 como número de porta de destino. Já para capturar pacotes HTTP, é necessário restringir os pacotes para aqueles que utilizem na camada inferior, protocolo TCP e que tenham 80 como número de porta de destino. O protocolo FTP
também utiliza protocolo TCP para transporte, mas tem como porta de destino o número 21.

	Nesse sentido, e utilizando as opções disponíveis para captura de pacotes da aplicação de diagnóstico \textit{tshark}, foram definidos os comandos necessários para se filtrar os pacotes capturados para cada um dos protocolos da camada de aplicação descritos acima, foram estes respectivamente:
\begin{lstlisting}[language=bash]

	tshark udp port 53

	tshark tcp port 80

	tshark tcp port 21

\end{lstlisting}

	Além dos comandos necessários para captura de pacotes, tem de se especificar um arquivo para direcionar a saída do comando executado, resultando nos seguintes comandos finais utilizados:
\begin{lstlisting}[language=bash]

	tshark udp port 53 > outputDNS.txt

	tshark tcp port 80 > outputHTTP.txt

	tshark tcp port 21 > outputFTP.txt

\end{lstlisting}

\section{Protocolo DNS}
\label{sec_dns}
		O DNS provê serviços de resolução de \textit{hosts} por meio da tradução de nomes de domínios em números de endereçamento IP. Como um mesmo domínio pode estar vinculado a vários endereços IP, esse serviço pode ser responsável também por distribuição de carga entre os \textit{hosts} de destino, alternando o endereço fornecido ao cliente quando uma solicitação de resolução de domínio é feita. Uma consulta DNS consiste basicamente de uma requisição enviada pelo cliente seguida de uma resposta devolvida pelo servidor. Nesse sentido, é interessante a utilização de um protocolo de transporte sem estabelecimento de conexão como o UDP para não acarretar em um atraso na requisição, como no caso do protocolo TCP que necessita de um \textit{handshake} de três vias.

		O serviço de DNS utiliza normalmente a porta 53, sendo esta observada na captura dos pacotes coletados.
\subsection{Dados coletados e Análise}
\label{sub_dns_dados}

Após a execução do comando de captura de pacotes especificado na seção anterior, foi aberto um novo terminal para teste de captura com o comando wget apontando para o nome de domínio do site tcpdump:
\begin{lstlisting}[language=bash]

	wget http://tcpdump.org/

\end{lstlisting}

	No arquivo de captura obteve-se os seguintes pacotes registrados:

\begin{lstlisting}

	0.000000    10.0.2.15 -> 201.10.1.4   DNS 75 Standard query 0x1e44
  A www.tcpdump.org
	0.000124    10.0.2.15 -> 201.10.1.4   DNS 75 Standard query 0xde66
  AAAA www.tcpdump.org
	0.350827   201.10.1.4 -> 10.0.2.15    DNS 131 Standard query response 0xde66
  AAAA 2001:4830:116e:2::6 AAAA 2607:f0b0:c:200::66
	0.356026   201.10.1.4 -> 10.0.2.15    DNS 107 Standard query response 0x1e44
  A 132.213.238.6 A 192.139.46.66

\end{lstlisting}

	Os pacotes enviados do cliente (IP 10.0.2.15) foram enviados ao \textit{host} de destino com IP 201.10.1.4 que respondeu às solicitações. É importante ressaltar que foram enviadas duas requisições do cliente, e as respostas do servidor DNS continham um identificador IPv6 no primeiro pacote recebido e um IPv4 no segundo. Como especificado na RFC 1886, que descreve sobre a adição de IPv6 com o registro AAAA em consultas DNS \cite{IETF}.

\section{Protocolo HTTP}
\label{sec_http}
O protocolo HTTP utilizado na camada de aplicação está vinculado à requsuição e ao carregamento de páginas \textit{web}, estas necessitam de tranferência confiável de dados para que nenhuma parte da informação presente na página seja perdida na rede. Nesse sentido, o protocolo HTTP utiliza na camada de transporte o protocolo TCP que disponibiliza em seus serviços transmissão confiável de dados. Esta é uma das características que tornam o TCP mais viável como protocolo de transporte para o HTTP na internet.

	Tendo como base o protocolo TCP, é necessário que ocorra então um \textit{handshake} em três vias para que mensagens HTTP possam transitar na rede. Além disso, menagens HTTP devem especificar a operação a ser executada (como GET, POST, PUT, etc) e a porta de destino utilizada que geralmente é a porta 80.
\subsection{Dados coletados e Análise}
\label{sub_http_dados}

	Após a execução do comando de captura de pacotes especificado na seção anterior, foi aberto um novo terminal para teste de captura com o comando wget apontando para o nome de domínio do site tcpdump, agora escutando por pacotes tcp na porta 80:

\begin{lstlisting}[language=bash]

	wget http://tcpdump.org/

\end{lstlisting}

	No arquivo de captura, dentre vários pacotes de tranferência de dados, foram selecionados alguns para demonstrar as operações executadas. São eles:

\begin{lstlisting}[language=bash]
	72.660377    10.0.2.15 -> 173.194.42.51 TCP 74 60075 > http
  [SYN] 	Seq=0 Win=14600 Len=0 MSS=1460 SACK_PERM=1 TSval=4846684 TSecr=0 WS=64
	72.814395 173.194.42.51 -> 10.0.2.15    TCP 60 http > 60075
  [SYN, ACK] Seq=0 Ack=1 Win=65535 Len=0 MSS=1460
	72.814427    10.0.2.15 -> 173.194.42.51 TCP 54 60075 > http
  [ACK] Seq=1 Ack=1 Win=14600 Len=0
	72.814984    10.0.2.15 -> 173.194.42.51 HTTP 168 GET / HTTP/1.1
	72.815162 173.194.42.51 -> 10.0.2.15    TCP 60 http > 60075
  [ACK] Seq=1 Ack=115 Win=65535 Len=0
	72.982628 173.194.42.51 -> 10.0.2.15    HTTP 550 HTTP/1.1 302 Found  (text/html)

     		. . . [finishing connection with host 173.194.42.51]

	73.141034    10.0.2.15 -> 173.194.42.56 TCP 74 46241 > http
  [SYN] Seq=0 Win=14600 Len=0 MSS=1460 SACK_PERM=1 TSval=4846804 TSecr=0 WS=64
	73.301732 173.194.42.56 -> 10.0.2.15    TCP 60 http > 46241
  [SYN, ACK] Seq=0 Ack=1 Win=65535 Len=0 MSS=1460
     73.301836    10.0.2.15 -> 173.194.42.56 TCP 54 46241 > http
  [ACK] Seq=1 Ack=1 Win=14600 Len=0

     		. . . [data transfer with host 173.194.42.56]

	73.700321 173.194.42.56 -> 10.0.2.15    HTTP 487 HTTP/1.1 200 OK  (text/html)
	73.700456    10.0.2.15 -> 173.194.42.56 TCP 54 46241 > http
  [ACK] Seq=154 Ack=20286 Win=41122 Len=0
	73.701066    10.0.2.15 -> 173.194.42.56 TCP 54 46241 > http
  [FIN, ACK] Seq=154 Ack=20286 Win=41122 Len=0
	73.701286 173.194.42.56 -> 10.0.2.15    TCP 60 http > 46241
  [ACK] Seq=20286 Ack=155 Win=65535 Len=0
	73.868597 173.194.42.56 -> 10.0.2.15    TCP 60 http > 46241
  [FIN, ACK] Seq=20286 Ack=155 Win=65535 Len=0
	73.868618    10.0.2.15 -> 173.194.42.56 TCP 54 46241 > http
  [ACK] Seq=155 Ack=20287 Win=41122 Len=0


\end{lstlisting}

	O primeiro bloco de pacotes mostrado é um \textit{handshake} em três vias do cliente IP 10.0.2.15 com o servidor IP 173.194.42.51, após o \textit{handshake}, o servidor notifica o cliente que encontrou a página requerida e o redireciona para outro servidor que poderá fornecer o pedido, com a flag HTTP 302. Esse comportamento pode indicar que o \textit{host} com IP 173.194.42.51 atua como um \textit{firewall} da rede.

	O segundo bloco de pacotes mostra novamente um \textit{handshake} em três vias do cliente IP 10.0.2.15, mas com o servidor IP 173.194.42.56 desta vez, e logo após se inicia a troca de mensagens para o recebimento da página requerida. Após a transferência, o servidor indica que a requisição está ok (HTTP 200), o cliente envia um ACK e após isso ocorre a finalização da conexão TCP, nos quatro últimos pacotes.

	O log completo de saída da captura de pacotes pode ser visualizado no apêndice A.

\section{Protocolo FTP}
\label{sec_ftp}

	O protocolo FTP é utilizado para tranferências de arquivos e, para isso, necessita de tranferência confiável de dados pela internet. Sendo assim, também faz uso do protocolo TCP para transporte de mensagens.

	Desse modo, é esperado para o FTP que ocorra um \textit{handshake} em três vias para estabelecimento de conexão com o servidor. Após o \textit{handshake}, ocorre a autenticação do usuário para que então ocorra o recebimento do arquivo.

\subsection{Dados coletados e Análise}
\label{sub_ftp_dados}
	Os passos descritos ateriormente podem ser observados nos pacotes recebidos na porta 21 monitorada quando se solicita um arquivo com o comando:

\begin{lstlisting}[language=bash]

	wget ftp://sppedtest.tele2.net/1KB.zip


\end{lstlisting}

	Os pacotes recebidos no arquivo de saída são:

\begin{lstlisting}[language=bash]
     0.000000    10.0.2.15 -> 90.130.70.73 TCP 74 40899 > ftp [SYN]
  Seq=0 Win=14600 Len=0 MSS=1460 SACK_PERM=1 TSval=8682778 TSecr=0 WS=64
     0.328906 90.130.70.73 -> 10.0.2.15    TCP 60 ftp > 40899 [SYN, ACK]
  Seq=0 Ack=1 Win=65535 Len=0 MSS=1460
     0.328945    10.0.2.15 -> 90.130.70.73 TCP 54 40899 > ftp [ACK]
  Seq=1 Ack=1 Win=14600 Len=0
     0.632970 90.130.70.73 -> 10.0.2.15    FTP 74 Response: 220 (vsFTPd 2.3.5)
     0.633001    10.0.2.15 -> 90.130.70.73 TCP 54 40899 > ftp [ACK] Seq=1 Ack=21
  Win=14600 Len=0
     0.633211    10.0.2.15 -> 90.130.70.73 FTP 70 Request: USER anonymous
     0.633472 90.130.70.73 -> 10.0.2.15    TCP 60 ftp > 40899 [ACK] Seq=21 Ack=17
  Win=65535 Len=0
     0.942161 90.130.70.73 -> 10.0.2.15    FTP 88 Response: 331 Please specify
  the password.
     0.942293    10.0.2.15 -> 90.130.70.73 FTP 67 Request: PASS -wget@
     0.942528 90.130.70.73 -> 10.0.2.15    TCP 60 ftp > 40899 [ACK] Seq=55 Ack=30
  Win=65535 Len=0
     1.276685 90.130.70.73 -> 10.0.2.15    FTP 77 Response: 230 Login successful.
     1.276878    10.0.2.15 -> 90.130.70.73 FTP 60 Request: SYST
     1.276996 90.130.70.73 -> 10.0.2.15    TCP 60 ftp > 40899 [ACK] Seq=78 Ack=36
  Win=65535 Len=0
     1.557551 90.130.70.73 -> 10.0.2.15    FTP 73 Response: 215 UNIX Type: L8
     1.557694    10.0.2.15 -> 90.130.70.73 FTP 59 Request: PWD
     1.557827 90.130.70.73 -> 10.0.2.15    TCP 60 ftp > 40899 [ACK] Seq=97 Ack=41
  Win=65535 Len=0
     1.867867 90.130.70.73 -> 10.0.2.15    FTP 63 Response: 257 "/"
     1.868044    10.0.2.15 -> 90.130.70.73 FTP 62 Request: TYPE I
     1.868320 90.130.70.73 -> 10.0.2.15    TCP 60 ftp > 40899 [ACK] Seq=106 Ack=49
  Win=65535 Len=0
     2.171201 90.130.70.73 -> 10.0.2.15    FTP 85 Response: 200 Switching to Binary mode.
     2.171366    10.0.2.15 -> 90.130.70.73 FTP 68 Request: SIZE 1KB.zip
     2.171679 90.130.70.73 -> 10.0.2.15    TCP 60 ftp > 40899 [ACK] Seq=137 Ack=63
  Win=65535 Len=0
     2.475659 90.130.70.73 -> 10.0.2.15    FTP 64 Response: 213 1024
     2.475784    10.0.2.15 -> 90.130.70.73 FTP 60 Request: PASV
     2.475938 90.130.70.73 -> 10.0.2.15    TCP 60 ftp > 40899 [ACK] Seq=147 Ack=69
  Win=65535 Len=0
     2.884872 90.130.70.73 -> 10.0.2.15    FTP 105 Response: 227 Entering
     Passive Mode (90,130,70,73,102,109)
     2.922799    10.0.2.15 -> 90.130.70.73 TCP 54 40899 > ftp [ACK] Seq=69 Ack=197
  Win=14600 Len=0
     3.198408    10.0.2.15 -> 90.130.70.73 FTP 68 Request: RETR 1KB.zip
     3.198535 90.130.70.73 -> 10.0.2.15    TCP 60 ftp > 40899 [ACK] Seq=197 Ack=83
  Win=65535 Len=0
     3.490530 90.130.70.73 -> 10.0.2.15    FTP 121 Response: 150
     Opening BINARY mode data connection for 1KB.zip (1024 bytes).
     3.490549    10.0.2.15 -> 90.130.70.73 TCP 54 40899 > ftp [ACK] Seq=83 Ack=264
  Win=14600 Len=0
     3.797609 90.130.70.73 -> 10.0.2.15    FTP 78 Response: 226 Transfer complete.
     3.797631    10.0.2.15 -> 90.130.70.73 TCP 54 40899 > ftp [ACK] Seq=83 Ack=288
  Win=14600 Len=0
     3.797766    10.0.2.15 -> 90.130.70.73 TCP 54 40899 > ftp [FIN, ACK] Seq=83 Ack=288
  Win=14600 Len=0
     3.797889 90.130.70.73 -> 10.0.2.15    TCP 60 ftp > 40899 [ACK] Seq=288 Ack=84
  Win=65535 Len=0
     4.119880 90.130.70.73 -> 10.0.2.15    FTP 94 Response: 500 OOPS:
  vsf_sysutil_recv_peek: no data
     4.119911    10.0.2.15 -> 90.130.70.73 TCP 54 40899 > ftp [RST] Seq=84 Win=0 Len=0

\end{lstlisting}

	Os pacotes recebidos evidenciam o \textit{handshake} entre o cliente IP 10.0.2.15 com o servidor IP 90.130.70.73 e é importante ressaltar que houve o processo de autorização (mesmo que de um usuário anônimo) para que a transferência ocorrece. Além disso, é possível observar também que o protocolo FTP não criptografa informações de usuário e senha, o que o torna pouco utilizado sem mecanismos de criptografia, como um protocolo ssl ou um ftp sob ssh.

%-------------------------------------------------------
\chapter{Ferramentas de Diagnóstico de Redes}
\label{chap_segundo}

\section{Monitoramento e Coleta de Pacotes}
\label{sec_segundo_monitoramento}

Para realizar o monitoramento e coleta de pacotes foi utilizado o sistema operacional \textit{Debian 7.7} e a aplicação \textit{tshark} para monitorar a troca de pacotes na rede.
Os \textit{outputs} da execução do \textit{tshark}, para cada uma das ferramentas listadas abaixo, são lançados para um arquivo com nome "output" + Nome da ferramenta + ".txt".

	A aplicação \textit{tshark} deve ser iniciada utilizando o comando listado para monitorar a interface de rede wireless:
\begin{lstlisting}[language=bash]
	sudo tshark -i wlan0
\end{lstlisting}

\begin{enumerate}[a)]
 \item Traceroute;
 \item Ping;
 \item Nmap;
\end{enumerate}

\subsection{Ping}
Para a realização da analise de pacotes enviados e recebidos pela ferramenta de diagnóstico Ping, foi utilizado a ferramenta WireShark. Esse experimento foi realizado utilizando-se um Mac OS X e as configurações necessárias para a execução do experimento estão listadas abaixo.

\begin{enumerate}
    \item Instalação do WireShark - https://www.wireshark.org/#download
    \item Instalação do XQuartz - Ferramenta que habilita o WireShark ser execultado no Mac OSX. - https://xquartz.macosforge.org/trac
\end{enumerate}


\section{Traceroute}
\label{sec_traceroute}
O traceroute é uma ferramenta de diagnostico que utiliza o campo \textit{TTL} (time-to-live) do protocolo IP para traçar o caminho tomado por um pacote dado um determinado \textit{host}. O programa tenta traçar uma rota lançando \textit{pacotes de prova} com um \textit{TTL} baixo, sendo o valor inicial igual a 1, e aguardando uma mensagem \textit{ICMP "time exceeded"}.  Este valor é acrescido de 1 sempre que alcança o próximo \textit{host} da rede até conseguir uma mensagem \textit{ICMP "port unreachable"}, que significa que atingiu o \textit{host} de destino.

\subsection{Dados coletados}
\label{sub_traceroute_dados}
Para realizar a captura de pacotes voltada para a aplicação \textit{traceroute} que utiliza o protocolo de transporte \textit{UDP} é necessário executar o comando:
\begin{lstlisting}[language=bash]
	sudo tshark -i wlan0 -f udp > outputTRACEROUTE.txt
\end{lstlisting}

	O \textit{Host} de destino se localiza no endereço \textbf{www.tcpdump.org}, então para traçar a rota executa-se em outro terminal o comando:

\begin{lstlisting}
	sudo traceroute www.tcpdump.org
\end{lstlisting}

	Esse comando especifico deve traçar a rota entre o \textit{host} local (192.168.1.8), através do \textit{gateway}, até o \textit{host} de destino.

Os pacotes coletados pela aplicação são:

\begin{lstlisting}
  1   0.000000  192.168.1.1 -> 239.255.255.250 SSDP 370 NOTIFY * HTTP/1.1
  2   0.211974  192.168.1.1 -> 239.255.255.250 SSDP 370 NOTIFY * HTTP/1.1
  3   5.672822  192.168.1.8 -> 192.168.1.1  DNS 75 Standard query 0x753d  A www.tcpdump.org
  4   5.672852  192.168.1.8 -> 192.168.1.1  DNS 75 Standard query 0x7d7a  AAAA www.tcpdump.org
  5  10.669333  192.168.1.1 -> 192.168.1.8  DNS 107 Standard query response 0x753d  A 132.213.238.6 A 192.139.46.66
  6  10.672603  192.168.1.8 -> 192.168.1.1  DNS 75 Standard query 0x753d  A www.tcpdump.org
  7  10.674831  192.168.1.1 -> 192.168.1.8  DNS 107 Standard query response 0x753d  A 192.139.46.66 A 132.213.238.6
  8  10.674878  192.168.1.8 -> 192.168.1.1  DNS 75 Standard query 0x7d7a  AAAA www.tcpdump.org
  9  10.677181  192.168.1.1 -> 192.168.1.8  DNS 75 Standard query response 0x7d7a Refused
 10  10.677502  192.168.1.8 -> 192.139.46.66 UDP 74 Source port: 44629  Destination port: 33434
 11  10.677533  192.168.1.8 -> 192.139.46.66 UDP 74 Source port: 42568  Destination port: 33435
 12  10.677555  192.168.1.8 -> 192.139.46.66 UDP 74 Source port: 40052  Destination port: 33436
 13  10.677573  192.168.1.8 -> 192.139.46.66 UDP 74 Source port: 56562  Destination port: 33437
 14  10.677593  192.168.1.8 -> 192.139.46.66 UDP 74 Source port: 52629  Destination port: 33438
 15  10.677610  192.168.1.8 -> 192.139.46.66 UDP 74 Source port: 58765  Destination port: 33439
 16  10.677628  192.168.1.8 -> 192.139.46.66 UDP 74 Source port: 34394  Destination port: 33440
 17  10.677646  192.168.1.8 -> 192.139.46.66 UDP 74 Source port: 46001  Destination port: 33441
 18  10.677664  192.168.1.8 -> 192.139.46.66 UDP 74 Source port: 35493  Destination port: 33442
 19  10.677682  192.168.1.8 -> 192.139.46.66 UDP 74 Source port: 34742  Destination port: 33443
 20  10.677700  192.168.1.8 -> 192.139.46.66 UDP 74 Source port: 54218  Destination port: 33444
 21  10.677721  192.168.1.8 -> 192.139.46.66 UDP 74 Source port: 54253  Destination port: 33445
 22  10.677739  192.168.1.8 -> 192.139.46.66 UDP 74 Source port: 52928  Destination port: 33446
 23  10.677757  192.168.1.8 -> 192.139.46.66 UDP 74 Source port: 38708  Destination port: 33447
 24  10.677775  192.168.1.8 -> 192.139.46.66 UDP 74 Source port: 58307  Destination port: 33448
 25  10.677989  192.168.1.8 -> 192.139.46.66 UDP 74 Source port: 46970  Destination port: 33449
 26  10.681903  192.168.1.8 -> 192.168.1.1  DNS 84 Standard query 0x70b0  PTR 1.1.168.192.in-addr.arpa
 27  10.692111  192.168.1.1 -> 192.168.1.8  DNS 118 Standard query response 0x70b0  PTR myrouter.domain.name
 28  10.692278  192.168.1.8 -> 192.139.46.66 UDP 74 Source port: 56630  Destination port: 33450
 29  10.692397  192.168.1.8 -> 192.139.46.66 UDP 74 Source port: 57671  Destination port: 33451
 30  10.692442  192.168.1.8 -> 192.139.46.66 UDP 74 Source port: 39032  Destination port: 33452
 31  10.693955  192.168.1.8 -> 192.168.1.1  DNS 87 Standard query 0xd2c4  PTR 71.126.184.179.in-addr.arpa
 32  10.697712  192.168.1.1 -> 192.168.1.8  DNS 126 Standard query response 0xd2c4  PTR gvt-b-sr05.bsa.gvt.net.br
 33  10.697852  192.168.1.8 -> 192.139.46.66 UDP 74 Source port: 51079  Destination port: 33453
 34  10.698327  192.168.1.8 -> 192.168.1.1  DNS 86 Standard query 0x9c14  PTR 231.8.205.177.in-addr.arpa
 35  10.701472  192.168.1.1 -> 192.168.1.8  DNS 136 Standard query response 0x9c14  PTR 177.205.8.231.static.adsl.gvt.net.br
 36  10.701648  192.168.1.8 -> 192.139.46.66 UDP 74 Source port: 48604  Destination port: 33454
 37  10.701706  192.168.1.8 -> 192.139.46.66 UDP 74 Source port: 36694  Destination port: 33455
 38  10.701750  192.168.1.8 -> 192.139.46.66 UDP 74 Source port: 44242  Destination port: 33456
 39  10.701841  192.168.1.8 -> 192.139.46.66 UDP 74 Source port: 60241  Destination port: 33457
 40  10.701884  192.168.1.8 -> 192.139.46.66 UDP 74 Source port: 47044  Destination port: 33458
 41  10.703049  192.168.1.8 -> 192.139.46.66 UDP 74 Source port: 42023  Destination port: 33459
 42  10.703105  192.168.1.8 -> 192.139.46.66 UDP 74 Source port: 35031  Destination port: 33460
 43  10.711614  192.168.1.8 -> 192.139.46.66 UDP 74 Source port: 53770  Destination port: 33461
 44  10.714361  192.168.1.8 -> 192.139.46.66 UDP 74 Source port: 38276  Destination port: 33462
 45  10.714417  192.168.1.8 -> 192.139.46.66 UDP 74 Source port: 45162  Destination port: 33463
 46  10.719628  192.168.1.8 -> 192.168.1.1  DNS 86 Standard query 0xd5d2  PTR 97.250.59.189.in-addr.arpa
 47  10.722438  192.168.1.1 -> 192.168.1.8  DNS 134 Standard query response 0xd5d2  PTR gvt-te-0-2-0-0.rc01.bsa.gvt.net.br
 48  10.722706  192.168.1.8 -> 192.168.1.1  DNS 87 Standard query 0x7c1d  PTR 101.250.59.189.in-addr.arpa
 49  10.725246  192.168.1.1 -> 192.168.1.8  DNS 135 Standard query response 0x7c1d  PTR gvt-te-0-2-0-0.rc02.bsa.gvt.net.br
 50  10.725584  192.168.1.8 -> 192.168.1.1  DNS 87 Standard query 0xefaf  PTR 117.251.59.189.in-addr.arpa
 51  10.727945  192.168.1.1 -> 192.168.1.8  DNS 135 Standard query response 0xefaf  PTR gvt-te-0-0-0-3.rc02.bhe.gvt.net.br
 52  10.728274  192.168.1.8 -> 192.168.1.1  DNS 86 Standard query 0xc70f  PTR 86.251.99.177.in-addr.arpa
 53  10.731398  192.168.1.1 -> 192.168.1.8  DNS 135 Standard query response 0xc70f  PTR gvt-te-0-5-0-10.rc01.gna.gvt.net.br
 54  10.731556  192.168.1.8 -> 192.139.46.66 UDP 74 Source port: 32822  Destination port: 33464
 55  10.731885  192.168.1.8 -> 192.168.1.1  DNS 87 Standard query 0x61bb  PTR 125.243.59.189.in-addr.arpa
 56  10.734487  192.168.1.1 -> 192.168.1.8  DNS 136 Standard query response 0x61bb  PTR gvt-te-0-3-0-11.rc02.spo.gvt.net.br
 57  10.734753  192.168.1.8 -> 192.168.1.1  DNS 87 Standard query 0xe8a7  PTR 106.251.99.177.in-addr.arpa
 58  10.737574  192.168.1.1 -> 192.168.1.8  DNS 135 Standard query response 0xe8a7  PTR gvt-te-0-4-0-0.rc01.bhe.gvt.net.br
 59  10.737832  192.168.1.8 -> 192.168.1.1  DNS 87 Standard query 0x2bd3  PTR 125.243.59.189.in-addr.arpa
 60  10.740158  192.168.1.1 -> 192.168.1.8  DNS 136 Standard query response 0x2bd3  PTR gvt-te-0-3-0-11.rc02.spo.gvt.net.br
 61  10.740295  192.168.1.8 -> 192.139.46.66 UDP 74 Source port: 51687  Destination port: 33465
 62  10.740352  192.168.1.8 -> 192.139.46.66 UDP 74 Source port: 40909  Destination port: 33466
 63  10.740397  192.168.1.8 -> 192.139.46.66 UDP 74 Source port: 33467  Destination port: 33467
 64  10.740595  192.168.1.8 -> 192.168.1.1  DNS 88 Standard query 0x3c6a  PTR 101.214.115.187.in-addr.arpa
 65  10.744628  192.168.1.1 -> 192.168.1.8  DNS 136 Standard query response 0x3c6a  PTR gvt-te-0-2-0-5.rc01.spo.gvt.net.br
 66  10.744824  192.168.1.8 -> 192.139.46.66 UDP 74 Source port: 45158  Destination port: 33468
 67  10.745083  192.168.1.8 -> 192.168.1.1  DNS 88 Standard query 0xdbc5  PTR 210.214.115.187.in-addr.arpa
 68  10.752758  192.168.1.1 -> 192.168.1.8  DNS 136 Standard query response 0xdbc5  PTR gvt-te-0-0-0-2.rt01.spo.gvt.net.br
 69  10.752922  192.168.1.8 -> 192.139.46.66 UDP 74 Source port: 56214  Destination port: 33469
 70  10.752980  192.168.1.8 -> 192.139.46.66 UDP 74 Source port: 40840  Destination port: 33470
 71  10.753023  192.168.1.8 -> 192.139.46.66 UDP 74 Source port: 33656  Destination port: 33471
 72  10.753173  192.168.1.8 -> 192.139.46.66 UDP 74 Source port: 48036  Destination port: 33472
 73  10.753222  192.168.1.8 -> 192.139.46.66 UDP 74 Source port: 39716  Destination port: 33473
 74  10.777941  192.168.1.8 -> 192.139.46.66 UDP 74 Source port: 46766  Destination port: 33474
 75  10.897850  192.168.1.8 -> 192.168.1.1  DNS 85 Standard query 0x09bd  PTR 33.246.69.67.in-addr.arpa
 76  10.900175  192.168.1.1 -> 192.168.1.8  DNS 141 Standard query response 0x09bd  PTR bx4-chicagodt_10GE0-4-0-0_peer.net.bell.ca
 77  10.900493  192.168.1.8 -> 192.168.1.1  DNS 84 Standard query 0xb8ab  PTR 5.26.208.64.in-addr.arpa
 78  10.902842  192.168.1.1 -> 192.168.1.8  DNS 137 Standard query response 0xb8ab  PTR TenGigabitEthernet8-2.ar5.GRU1.gblx.net
 79  10.903004  192.168.1.8 -> 192.139.46.66 UDP 74 Source port: 53646  Destination port: 33475
 80  10.903068  192.168.1.8 -> 192.139.46.66 UDP 74 Source port: 45409  Destination port: 33476
 81  11.080042  192.168.1.8 -> 192.139.46.66 UDP 74 Source port: 39235  Destination port: 33477
 82  11.257463  192.168.1.8 -> 192.139.46.66 UDP 74 Source port: 58333  Destination port: 33478
 83  16.262872  192.168.1.8 -> 192.168.1.1  DNS 85 Standard query 0xd602  PTR 26.48.35.207.in-addr.arpa
 84  16.265138  192.168.1.1 -> 192.168.1.8  DNS 85 Standard query response 0xd602 No such name
 85  16.265314  192.168.1.8 -> 192.139.46.66 UDP 74 Source port: 44471  Destination port: 33479
 86  16.265350  192.168.1.8 -> 192.139.46.66 UDP 74 Source port: 41608  Destination port: 33480
 87  16.265378  192.168.1.8 -> 192.139.46.66 UDP 74 Source port: 57472  Destination port: 33481
 88  16.265397  192.168.1.8 -> 192.139.46.66 UDP 74 Source port: 43896  Destination port: 33482
 89  16.265416  192.168.1.8 -> 192.139.46.66 UDP 74 Source port: 59735  Destination port: 33483
 90  16.265436  192.168.1.8 -> 192.139.46.66 UDP 74 Source port: 43869  Destination port: 33484
 91  16.265453  192.168.1.8 -> 192.139.46.66 UDP 74 Source port: 52857  Destination port: 33485
 92  16.265469  192.168.1.8 -> 192.139.46.66 UDP 74 Source port: 51565  Destination port: 33486
 93  16.265485  192.168.1.8 -> 192.139.46.66 UDP 74 Source port: 55025  Destination port: 33487
 94  16.265501  192.168.1.8 -> 192.139.46.66 UDP 74 Source port: 36206  Destination port: 33488
 95  16.265526  192.168.1.8 -> 192.139.46.66 UDP 74 Source port: 56226  Destination port: 33489
 96  16.265544  192.168.1.8 -> 192.139.46.66 UDP 74 Source port: 60352  Destination port: 33490
 97  16.265559  192.168.1.8 -> 192.139.46.66 UDP 74 Source port: 59909  Destination port: 33491
 98  16.265697  192.168.1.8 -> 192.139.46.66 UDP 74 Source port: 58043  Destination port: 33492
 99  16.265769  192.168.1.8 -> 192.139.46.66 UDP 74 Source port: 40247  Destination port: 33493
100  16.265787  192.168.1.8 -> 192.139.46.66 UDP 74 Source port: 33991  Destination port: 33494
101  16.490890  192.168.1.8 -> 192.168.1.1  DNS 85 Standard query 0xb8ab  PTR 26.48.35.207.in-addr.arpa
102  16.493673  192.168.1.1 -> 192.168.1.8  DNS 85 Standard query response 0xb8ab No such name
103  16.493822  192.168.1.8 -> 192.139.46.66 UDP 74 Source port: 47618  Destination port: 33495
104  16.493918  192.168.1.8 -> 192.139.46.66 UDP 74 Source port: 48875  Destination port: 33496
105  16.493953  192.168.1.8 -> 192.139.46.66 UDP 74 Source port: 41777  Destination port: 33497
106  16.504453  192.168.1.8 -> 192.168.1.1  DNS 86 Standard query 0x8f79  PTR 86.254.87.209.in-addr.arpa
107  16.506888  192.168.1.1 -> 192.168.1.8  DNS 132 Standard query response 0x8f79  PTR storm.riddler.batcave.credil.org
108  16.507074  192.168.1.8 -> 192.139.46.66 UDP 74 Source port: 38466  Destination port: 33498
109  21.513446  192.168.1.8 -> 192.139.46.66 UDP 74 Source port: 57657  Destination port: 33499
110  21.513514  192.168.1.8 -> 192.139.46.66 UDP 74 Source port: 38057  Destination port: 33500
111  21.513556  192.168.1.8 -> 192.139.46.66 UDP 74 Source port: 53629  Destination port: 33501
112  21.513595  192.168.1.8 -> 192.139.46.66 UDP 74 Source port: 58680  Destination port: 33502
113  21.513634  192.168.1.8 -> 192.139.46.66 UDP 74 Source port: 39813  Destination port: 33503
114  21.513673  192.168.1.8 -> 192.139.46.66 UDP 74 Source port: 43537  Destination port: 33504
115  21.513711  192.168.1.8 -> 192.139.46.66 UDP 74 Source port: 47426  Destination port: 33505
116  21.513748  192.168.1.8 -> 192.139.46.66 UDP 74 Source port: 40317  Destination port: 33506
117  21.514336  192.168.1.8 -> 192.139.46.66 UDP 74 Source port: 44399  Destination port: 33507
118  21.514376  192.168.1.8 -> 192.139.46.66 UDP 74 Source port: 48149  Destination port: 33508
119  21.514547  192.168.1.8 -> 192.139.46.66 UDP 74 Source port: 46485  Destination port: 33509
120  21.514588  192.168.1.8 -> 192.139.46.66 UDP 74 Source port: 40011  Destination port: 33510
121  21.514624  192.168.1.8 -> 192.139.46.66 UDP 74 Source port: 51647  Destination port: 33511
122  21.514663  192.168.1.8 -> 192.139.46.66 UDP 74 Source port: 36191  Destination port: 33512
123  21.514706  192.168.1.8 -> 192.139.46.66 UDP 74 Source port: 34635  Destination port: 33513
124  21.514824  192.168.1.8 -> 192.139.46.66 UDP 74 Source port: 50736  Destination port: 33514
125  26.516796  192.168.1.8 -> 192.139.46.66 UDP 74 Source port: 46273  Destination port: 33515
126  26.516846  192.168.1.8 -> 192.139.46.66 UDP 74 Source port: 40401  Destination port: 33516
127  26.516876  192.168.1.8 -> 192.139.46.66 UDP 74 Source port: 60115  Destination port: 33517
128  26.516904  192.168.1.8 -> 192.139.46.66 UDP 74 Source port: 46207  Destination port: 33518
129  26.516932  192.168.1.8 -> 192.139.46.66 UDP 74 Source port: 44778  Destination port: 33519
130  26.516959  192.168.1.8 -> 192.139.46.66 UDP 74 Source port: 45370  Destination port: 33520
131  26.516985  192.168.1.8 -> 192.139.46.66 UDP 74 Source port: 35131  Destination port: 33521
132  26.517009  192.168.1.8 -> 192.139.46.66 UDP 74 Source port: 40772  Destination port: 33522
133  26.517032  192.168.1.8 -> 192.139.46.66 UDP 74 Source port: 44717  Destination port: 33523

\end{lstlisting}

\subsection{Análise}
\label{sub_traceroute_analise}
É possível observar que a troca de pacotes acontece através do protocolo UDP e também que a porta de comunicação e o campo \textit{TTL} são incrementados, até que receba uma \textit{ICMP ["port unreachable"]} (não aparece no log) indicando que chegou ao destino.
Não foi observado perdas de pacote, já que não houve troca de mensagem \textit{ICMP ["time exceeded"]}. Para a simulação realizada o endereço IP de origem 192.168.1.8 e destino 192.139.46.66.


\section{Ping}
\label{sec_ping}
O Ping é um software utilitário de administração de rede de computadores utilizada para testar a acessibilidade de um host em uma rede IP (Internet Protocol) e para medir o tempo de ida e volta de mensagens enviadas a partir do host de origem para o computador de destino.

O Ping opera enviando uma mensagem baseada no ICMP (Internet Control Message Protocol), pacotes de solicitação de eco para o host de destino. Ele mede o tempo de ida e volta de transmissão para recepção, erros e perda de pacotes de relatório. Os resultados do teste geralmente incluem um resumo estatístico dos pacotes de resposta recebida, incluindo o mínimo, o máximo e os tempos médios de ida e volta.

\subsection{Dados coletados}
\label{sub_ping_dados}
Para a coleta de pacotes da ferramenta ping, foi inicializada a análise da rede pelo WireShark. Em seguinda iniciou-se a ferramenta ping, conforme descrito anteriormente, execultando o comando:

\begin{lstlisting}[language=bash]
ping www.google.com
\end{lstlisting}

O resultado do comando acima é uma sequência de requisições do cliente(localhost) para o servidor, nesse caso: www.google.com.

\begin{lstlisting}[language=bash]
PING www.google.com (189.6.76.187): 56 data bytes
64 bytes from 189.6.76.187: icmp_seq=0 ttl=61 time=12.604 ms
64 bytes from 189.6.76.187: icmp_seq=1 ttl=61 time=9.938 ms
64 bytes from 189.6.76.187: icmp_seq=2 ttl=61 time=15.585 ms
64 bytes from 189.6.76.187: icmp_seq=3 ttl=61 time=13.192 ms
64 bytes from 189.6.76.187: icmp_seq=4 ttl=61 time=13.233 ms
64 bytes from 189.6.76.187: icmp_seq=5 ttl=61 time=13.645 ms

--- www.google.com ping statistics ---
6 packets transmitted, 6 packets received, 0.0% packet loss
round-trip min/avg/max/stddev = 9.938/13.033/15.585/1.669 ms
\end{lstlisting}

Após o inicio da ferramenta ping, foi visto os pacotes de requisição e de resposta respectivamente conforme abaixo.

\begin{lstlisting}[language=bash]
Ethernet II, Src: Apple_c5:5c:c8 (3c:15:c2:c5:5c:c8), Dst: Technico_5b:59:b6
(58:23:8c:5b:59:b6)
  Destination: Technico_5b:59:b6 (58:23:8c:5b:59:b6)
  Source: Apple_c5:5c:c8 (3c:15:c2:c5:5c:c8)
  Type: IP (0x0800)
Internet Protocol Version 4, Src: 192.168.0.14 (192.168.0.14), Dst: 189.6.76.187
(189.6.76.187)
  Version: 4
  Header Length: 20 bytes
  Differentiated Services Field: 0x00 (DSCP 0x00: Default; ECN: 0x00: Not-ECT
  (Not ECN-Capable Transport))
  Total Length: 84
  Identification: 0x4883 (18563)
  Flags: 0x00
  Fragment offset: 0
  Time to live: 64
  Protocol: ICMP (1)
  Header checksum: 0x67ae [validation disabled]
  Source: 192.168.0.14 (192.168.0.14)
  Destination: 189.6.76.187 (189.6.76.187)
Internet Control Message Protocol
  Type: 8 (Echo (ping) request)
  Code: 0
  Checksum: 0x151e [correct]
  Identifier (BE): 53676 (0xd1ac)
  Identifier (LE): 44241 (0xacd1)
  Sequence number (BE): 9 (0x0009)
  Sequence number (LE): 2304 (0x0900)
\end{lstlisting}

\begin{lstlisting}[language=bash]
Ethernet II, Src: Technico_5b:59:b6 (58:23:8c:5b:59:b6), Dst: Apple_c5:5c:c8
(3c:15:c2:c5:5c:c8)
  Destination: Apple_c5:5c:c8 (3c:15:c2:c5:5c:c8)
  Source: Technico_5b:59:b6 (58:23:8c:5b:59:b6)
  Type: IP (0x0800)
Internet Protocol Version 4, Src: 189.6.76.187 (189.6.76.187), Dst: 192.168.0.14
(192.168.0.14)
  Version: 4
  Header Length: 20 bytes
  Differentiated Services Field: 0x00 (DSCP 0x00: Default; ECN: 0x00: Not-ECT
  (Not ECN-Capable Transport))
  Total Length: 84
  Identification: 0x00008d46
  Flags: 0x00
  Fragment offset: 0
  Time to live: 61
  Protocol: ICMP (1)
  Header checksum: 0x67ae [validation disabled]
  Source: 189.6.76.187 (189.6.76.187)
  Destination: 192.168.0.14 (192.168.0.14)
Internet Control Message Protocol
  Type: 0 (Echo (ping) reply)
  Code: 0
  Checksum: 0x1d1e [correct]
  Identifier (BE): 53676 (0xd1ac)
  Identifier (LE): 44241 (0xacd1)
  Sequence number (BE): 9 (0x0009)
  Sequence number (LE): 2304 (0x0900)
  Request frame: 207
  Timestamp from icmp data: Oct 25, 2015 21:22:10.328175000 BRST
  Timestamp from icmp data (relative): 0.013197000 seconds

\end{lstlisting}

\subsection{Análise}
\label{sub_ping_analise}
Uma vez que as requisições se repetem durante a execução do Ping, nesse relatório foram analisados uma requisição e uma resposta.

Conforme dito na sessão \ref{sec_ping} o Ping utiliza o ICMP, nos pacotes acima nota-se a definição do protocolo que será utilizado para a comunicação entre os hosts.

Além dos campos de destino e origem (destination e source) que definem a origem e o destino dos pacotes, o campo de \textit{"Timestamp"} onde é calculado o real valor de tempo do transporte do pacote. Nos pacotes analisados tem-se que o tempo de ping foi de 0.013197000 seconds, transformando para ms tem-se 13.197 ms, ou seja, os pacotes analisados nesse experimento são referentes à quarta requisão de Ping.

Ao final da execução do Ping, é exibido os dados estatísticos.

\begin{lstlisting}[language=bash]
--- www.google.com ping statistics ---
6 packets transmitted, 6 packets received, 0.0% packet loss
round-trip min/avg/max/stddev = 9.938/13.033/15.585/1.669 ms
\end{lstlisting}

Nessa execução foram transmitidos 6 pacotes, sem perdas. O tempo minimo foi de 9.938 ms, o máximo de 15.585ms, a média de 13.033 e o desvio padrão do tempo médio de 1.669ms conforme visto no código acima.

\section{Nmap}
\label{sec_nmap}
O \textit{Nmap} é um software livre utilizado para varrer as portas de um \textit{host} visando encontrar problemas de segurança. É uma potente ferramenta devido a simplicidade e opções de utilização. Utiliza o protocolo \textbf{IP} em estado bruto (raw).


\subsection{Dados coletados}
\label{sub_nmap_dados}
Aqui vão os dados coletados para a ferramenta nmap.

\subsection{Análise}
\label{sub_nmap_analise}
Aqui vai a análise dos dados coletados para a ferramenta nmap.

% Referencias bibliograficas
\begin{thebibliography}{99}
\bibitem{Stewart} STEWART, James. {\sl C\'alculo.} Vol. 1. S\~ao Paulo: Pioneira, 2006.
\bibitem{IETF} IETF, Internet Engineering Task Force, RFC 1886, \url{http://tools.ietf.org/html/rfc1886}, 1995.

\bibitem{LaTeX}
\url{http://latexbr.blogspot.com}
\end{thebibliography}

\addcontentsline{toc}{chapter}{Refer\^encias Bibliogr\'aficas}

\end{document}  %Fim do documento
