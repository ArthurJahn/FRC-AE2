
\documentclass[a4paper]{report} %padrao letterpaper, 10pt
\usepackage[utf8]{inputenc}
\usepackage[brazil]{babel}
\usepackage{amsfonts,amssymb,graphicx,enumerate}
\usepackage[centertags]{amsmath}
% Configuracoes de pagina
\usepackage[lmargin=3cm,rmargin=3cm,tmargin=3cm,bmargin=3cm]{geometry}
% Layout da pagina
\usepackage{hyperref}
\hypersetup{pdfpagelayout=SinglePage, % ou TwoPageLeft
    colorlinks=true,
    pdftitle={Relat\'orio de asuhaush Redes},
    pdfauthor={\'Aulus Diniz, Arthur Jahn}}

%*******************************************************


\title{Fundamentos de Redes de Computadores: Camada de transporte}
\author{Arthur Jahn, Áulus Diniz}
\date{26 de Outubro de 2015}    %para ocultar a data digite: \date{ }
%*******************************************************
\begin{document}    %Inicio do documento
\maketitle  %cria o titulo na capa

\tableofcontents %Sumario
%-------------------------------------------------------
\chapter{Protocolos de Camada de Aplicação}
\label{chap_primeiro} %rotulo que pode ser usado em referencia cruzada

\section{Monitoramento e Coleta de Pacotes}
\label{sec_primeiro_monitoramento}

Aqui deve vir o processo de preparação para o monitoramento e captura de pacotes para os protocolos da camada
de aplicação. Sendo estes:  
\begin{enumerate}[a)]
 \item HTTP;
 \item FTP;
 \item DNS;
\end{enumerate}

\section{Protocolo HTTP}
\label{sec_http}
Aqui vai algo sobre o protocolo http.
\subsection{Dados coletados}
\label{sub_http_dados}
Aqui vão os dados coletados para o protocolo HTTP.

\subsection{Análise}
\label{sub_http_analise}
Aqui vai a análise dos dados coletados para o protocolo HTTP.

\section{Protocolo DNS}
\label{sec_dns}
Aqui vai algo sobre o protocolo dns.
\subsection{Dados coletados}
\label{sub_dns_dados}
Aqui vão os dados coletados para o protocolo DNS.

\subsection{Análise}
\label{sub_dns_analise}
Aqui vai a análise dos dados coletados para o protocolo DNS.

\section{Protocolo FTP}
\label{sec_ftp}
Aqui vai algo sobre o protocolo ftp.
\subsection{Dados coletados}
\label{sub_ftp_dados}
Aqui vão os dados coletados para o protocolo FTP.

\subsection{Análise}
\label{sub_ftp_analise}
Aqui vai a análise dos dados coletados para o protocolo FTP.

%-------------------------------------------------------
\chapter{Segundo capítulo}
\label{chap_segundo}

\section{Protocolos}
\label{sec_segundo_protocolos}

\section{Dados coletados}
\label{sec_segundo}

\section{Análise}
\label{sec_segundo}

\subsection{Subseção}
\label{subsec_nome}



% Referencias bibliograficas
\begin{thebibliography}{99}
\bibitem{Stewart} STEWART, James. {\sl C\'alculo.} Vol. 1. S\~ao Paulo: Pioneira, 2006.

\bibitem{Oetiker}
OETIKER, Tobias. Et. Al. {\sl Introdu\c c\~ao ao {\LaTeXe}.}
\url{http://mirrors.ctan.org/info/lshort/portuguese-BR/lshortBR.pdf},2001.

\bibitem{Tantau}
TANTAU, Till. {\sl The TikZ and PGF Packages.}
\url{http://ctan.tche.br/graphics/pgf/base/doc/generic/pgf/pgfmanual.pdf}, 2010.

\bibitem{LaTeX}
\url{http://latexbr.blogspot.com}
\end{thebibliography}

\addcontentsline{toc}{chapter}{Refer\^encias Bibliogr\'aficas}

\end{document}  %Fim do documento